\documentclass[10pt,a4paper]{article}
\usepackage[utf8]{inputenc}
\usepackage{amsmath}
\usepackage{amsfonts}
\usepackage{amssymb}
\usepackage{makeidx}
\title{% 
Should Chemistry be a prerequisite to pursue Civil Engineering? \\
\large Correction in my App engine URL-\textbf{peterwauyo12.appspot.com} }
\author{WAUYO PETER}
\date{\today}

\begin{document}

\maketitle

\paragraph{1 Introduction}
\begin{flushleft}
Chemistry is the discipline which deals with forming new substances from different basic materials. For this reason it can be said that chemistry is inextricably linked with civil engineering - although this link is not as obvious as that to physics.
\end{flushleft}

\paragraph{1.1 Background}
\begin{flushleft}
Civil engineering is a professional engineering discipline that deals with the design, construction, and maintenance of the physical and naturally built environment, including works like roads, bridges, canals, dams, airports, sewerage systems, pipelines and railways
\newline
\newline
Civil engineers tend to prefer to care more about the properties of materials as they are than about the way they change and react with each other, but the materials don't know this, and they quite inconsiderately change and react with each other and with the environment. Civil engineers need to understand the chemistry of corrosion, the chemistry of concrete, and so on. You usually won't find civil engineers occupied with the finer points of the latest in synthesis technology or anything like that, but even a civil engineer who does not specialize in anything that sounds like it involves chemistry needs to understand at least some chemical processes.
\end{flushleft}

\paragraph{1.2 Problem Statement}
\begin{flushleft}
This project will examine the impact of a civil engineering student’s chemistry background on their academic performance and engineering skills.
\end{flushleft}

\paragraph{1.3 Objectives}

\paragraph{1.3.1 Main Objective}
\begin{flushleft}
The main goal of this project is to determine whether one needs to be good at chemistry to excel in civil engineering. 
\end{flushleft}

\paragraph{1.3.2 Specific Objectives}
\begin{flushleft}
\begin{itemize}
	\item To collect all the data necessary to aid our research.
	\item To perform a thorough analysis on the collected data.
	\item To come up with a conclusion from the data analysis. 
\end{itemize}
\end{flushleft}

\paragraph{1.4 Scope}
\begin{flushleft}
This research is aimed at civil engineering students at higher institutions of learning such as Makerere University.
\end{flushleft}

\paragraph{1.5 Research Significance}
\begin{flushleft}
 This study is important because it aims at improving the civil engineering curriculum at higher institutions of learning.
\end{flushleft}

\paragraph{2 Literature review}
\begin{flushleft}
We looked at [1]Makerere University’s B.Sc. (Civil Engineering ) program, and for a candidate to be admitted, he/she must have: At least a subsidiary pass in chemistry in the Uganda Advanced Certificate of Education (UACE) or its equivalent and at least two principle passes at the same sitting in UACE in any of the following subjects: -Mathematics, Physics, Chemistry, Technical Drawing.
\end{flushleft}

\paragraph{3 Methodology}
\begin{flushleft}
The proposed methodology consists of two phases, data collection and data analysis. Data will be collected using ODK Collect, which will later on be uploaded to the ODK aggregate server to carry out all the required analysis. Different kinds of data (including images and GPS coordinates) will be collected, these include:
\begin{itemize}
 \item Students name, recent photo and place of residence (including GPS coordinates)
 \item Employee status
 \item Engineering level (beginner, intermediate and none)
 \item UACE Chemistry result
 \item Current performance at the university (CGPA)
\end{itemize} 
\end{flushleft}

\end{document}